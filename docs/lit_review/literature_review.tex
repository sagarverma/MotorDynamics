\documentclass{article}
\usepackage[utf8]{inputenc}

\usepackage{amsmath}
\usepackage{hyperref}
\hypersetup{
    colorlinks=true,
    linkcolor=blue,
    filecolor=magenta,
    urlcolor=cyan,
}


\title{Literature Review: Machine Learning Applied to Dynamic Physial System.}
\date{August 2018}

\begin{document}

\maketitle

\section{Abstract}

\section{Background}
\subsection{Modeling of physical systems}
\begin{enumerate}
  \item Traditional work in modeling physical systems \\
  Automated Design of Complex Dynamic Systems \\

  \item Data driven design \\
  Theory-Guided Data Science: A New Paradigmfor Scientific Discovery from Data \\
  \begin{enumerate}
    \item Machine learning based approach \\
    \item Deep learning based approach \\
      Towards a Hybrid Approach to Physical ProcessModeling \\
      Deep learning for universal linear embeddings of nonlinear dynamics \\
      Nonlinear Systems Identification Using Deep Dynamic Neural Networks \\
      Analyzing Inverse Problems withInvertible Neural Networks \\
      Deep Hidden Physics Models:  Deep Learning ofNonlinear Partial Differential Equations \\
      How Can Physics Inform Deep Learning Methods inScientific Problems?: Recent Progress andFuture Prospects \\
      Learning New Physics from a Machine \\
      Nanophotonic Particle Simulation and Inverse DesignUsing Artificial Neural Networks \\
      Particle Track Reconstruction with Deep Learning \\
      Neural Message Passing for Jet Physics \\
      Physics-guided Neural Networks (PGNN):An Application in Lake Temperature Modeling \\
    \item Reinforcement learning based approach \\
      Large-Scale Study of Curiosity-Driven Learning \\
      DeepMimic: Example-Guided Deep Reinforcement Learningof Physics-Based Character Skills \\
    \item Adversarial learning based approach \\
      Tips and Tricks for Training GANs with PhysicsConstraints \\
      Adversarial learning to eliminate systematic errors:a case study in High Energy Physics \\
  \end{enumerate}

\end{enumerate}

\subsection{Solving PDEs}
Solving differential equations with unknownconstitutive relations as recurrent neural networks \\
\subsection{Non-linear control}
\subsection{Motor control}
\subsection{Time series}

\end{document}
