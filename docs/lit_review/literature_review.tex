\documentclass[conference]{IEEEtran}
\IEEEoverridecommandlockouts
% The preceding line is only needed to identify funding in the first footnote. If that is unneeded, please comment it out.
\usepackage{cite}
\usepackage{amsmath,amssymb,amsfonts}
\usepackage{algorithmic}
\usepackage{graphicx}
\usepackage{textcomp}
\usepackage{booktabs, times, epsfig, graphicx, amsmath, amssymb,url, multirow,comment,commath}


\begin{document}

\title{Literature Review: Machine Learning Applied to Dynamic Physial System}
\date{August 2018}

\maketitle

\section{Abstract}

\section{Background}
\subsection{Modeling of physical systems}
\begin{enumerate}
  \item Traditional work in modeling physical systems \\
  Automated Design of Complex Dynamic Systems \cite{hermans2014automated} \\

  \item Data driven design \\
  Theory-Guided Data Science: A New Paradigmfor Scientific Discovery from Data \cite{karpatne2017theory-guided}\\
  \begin{enumerate}
    \item Machine learning based approach \\
      Data-Driven Discovery of Governing Physical Laws and Their Parameteric Dependencies in Engineering, Physics and Biology \cite{kutz2017datadriven} \\
      Data-driven discovery of partial differential equations \cite{rudy2017datadriven} \\
      Discovering governing equations from data: Sparse identification of nonlinear dynamical systems \cite{Brunton3932} \\
    \item Deep learning based approach \\
      Towards a Hybrid Approach to Physical ProcessModeling \\
      Deep learning for universal linear embeddings of nonlinear dynamics \cite{lusch2017deep} \\
      Nonlinear Systems Identification Using Deep Dynamic Neural Networks \cite{ogunmolu2016nonlinear}\\
      Analyzing Inverse Problems withInvertible Neural Networks \cite{ardizzone2018analyzing}\\
      Deep Hidden Physics Models:  Deep Learning ofNonlinear Partial Differential Equations \cite{raissi2018deep} \\
      How Can Physics Inform Deep Learning Methods inScientific Problems?: Recent Progress andFuture Prospects \\
      Learning New Physics from a Machine \cite{d'agnolo2018learning}\\
      Nanophotonic Particle Simulation and Inverse DesignUsing Artificial Neural Networks \\
      Particle Track Reconstruction with Deep Learning \\
      Neural Message Passing for Jet Physics \\
      Physics-guided Neural Networks (PGNN):An Application in Lake Temperature Modeling \cite{karpatne2017physics-guided} \\
    \item Reinforcement learning based approach \\
      Large-Scale Study of Curiosity-Driven Learning \cite{burda2018large-scale}\\
      DeepMimic: Example-Guided Deep Reinforcement Learningof Physics-Based Character Skills \cite{peng2018deepmimic} \\
    \item Adversarial learning based approach \\
      Tips and Tricks for Training GANs with PhysicsConstraints \\
      Adversarial learning to eliminate systematic errors:a case study in High Energy Physics \\
  \end{enumerate}

\end{enumerate}

\subsection{Solving PDEs}
Solving differential equations with unknownconstitutive relations as recurrent neural networks \\

\subsection{Non-linear control}
Adaptive Inverse Control of Linear and Nonlinear Systems Using Dynamic Neural Networks \cite{plett2003adaptive} \\
Nonlinear System Control Using Neural Networks \\
Feedback-Linearization-Based Neural Adaptive Control for Unknown Nonaffine Nonlinear Discrete-Time Systems \\
A Novel Neural Approximate Inverse Control for Unknown Nonlinear Discrete Dynamical Systems \cite{deng2005a} \\
Intelligent Control Using Neural Networks and Multiple Models \cite{fu2008intelligent} \\
Dynamic Power Conditioning Method of Microgrid Via Adaptive Inverse Control \cite{li2015dynamic} \\
Discrete-time neuroadaptive control using dynamic state feedback with application to vehicle motion control for intelligent vehicle highway systems \cite{kumarawadu2010discrete-time} \\
Identification and Adaptive Control of Dynamic Nonlinear Systems Using Sigmoid Diagonal Recurrent Neural Network \cite{aboueldahab2011identification} \\

\subsection{Motor control}
\subsection{Time series}


\bibliographystyle{IEEEtran}
\bibliography{literature_review}

\end{document}
